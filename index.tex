\documentclass[]{elsarticle} %review=doublespace preprint=single 5p=2 column
%%% Begin My package additions %%%%%%%%%%%%%%%%%%%
\usepackage[hyphens]{url}



\usepackage{lineno} % add
\providecommand{\tightlist}{%
  \setlength{\itemsep}{0pt}\setlength{\parskip}{0pt}}

\usepackage{graphicx}
\usepackage{booktabs} % book-quality tables
%%%%%%%%%%%%%%%% end my additions to header

\usepackage[T1]{fontenc}
\usepackage{lmodern}
\usepackage{amssymb,amsmath}
\usepackage{ifxetex,ifluatex}
\usepackage{fixltx2e} % provides \textsubscript
% use upquote if available, for straight quotes in verbatim environments
\IfFileExists{upquote.sty}{\usepackage{upquote}}{}
\ifnum 0\ifxetex 1\fi\ifluatex 1\fi=0 % if pdftex
  \usepackage[utf8]{inputenc}
\else % if luatex or xelatex
  \usepackage{fontspec}
  \ifxetex
    \usepackage{xltxtra,xunicode}
  \fi
  \defaultfontfeatures{Mapping=tex-text,Scale=MatchLowercase}
  \newcommand{\euro}{€}
\fi
% use microtype if available
\IfFileExists{microtype.sty}{\usepackage{microtype}}{}
\bibliographystyle{elsarticle-harv}
\usepackage{longtable}
\ifxetex
  \usepackage[setpagesize=false, % page size defined by xetex
              unicode=false, % unicode breaks when used with xetex
              xetex]{hyperref}
\else
  \usepackage[unicode=true]{hyperref}
\fi
\hypersetup{breaklinks=true,
            bookmarks=true,
            pdfauthor={},
            pdftitle={Growing Neuroconductor},
            colorlinks=false,
            urlcolor=blue,
            linkcolor=magenta,
            pdfborder={0 0 0}}
\urlstyle{same}  % don't use monospace font for urls

\setcounter{secnumdepth}{5}
% Pandoc toggle for numbering sections (defaults to be off)
% Pandoc header



\begin{document}
\begin{frontmatter}

  \title{Growing Neuroconductor}
    
  \begin{abstract}
  Over the past 5 years, Neuroconductor has centralized the packages of medical image analysis in the R community. As a leading open source power tool to work with messy
  data, it is taught in countless courses and workshops around the world. In
  biomedical research alone, OpenRefine is cited in hundreds of scientific articles, in fields such as genomics {[}1, 2, 3, 4{]}, Alzheimer's disease {[}5, 6, 7{]},
  infectious diseases {[}8, 9, 10{]}, oncology {[}11, 12, 13, 14{]} and clinical data management {[}15, 16, 17, 18{]}. To keep OpenRefine thriving in the next ten years,
  we need to grow its contributor community and undertake fundamental improvements to its architecture.
  \end{abstract}
  
 \end{frontmatter}

\hypertarget{goals}{%
\section{Goals}\label{goals}}

The main objectives of this proposal are:

\begin{enumerate}
\def\labelenumi{\arabic{enumi}.}
\tightlist
\item
  increase the number of packages and community of Neuroconductors contributors and developers by reaching out to seasoned users and helping them get involved more closely in the project.
\item
  improve the stability of the current packages in Neuroconductor.
\item
  refactor the core architecture of the Neuroconductor backend platform to handle more packages and incorporate user workflows.
\end{enumerate}

Our goals will be reached if:

\begin{enumerate}
\def\labelenumi{\arabic{enumi}.}
\tightlist
\item
  We add at least 10 new packages over the course of the project and get contributions from 20 existing contributors. Contributions will be defined as release updates to packages through developers GitHub repositories or submission additional packages to release.
\item
  Each package has a vignette (a tutorial/long-form documentation) and the average code coverage of all packages is above 30\% (what is it now).
\item
  Neuroconductor can scale to 200 packages (currently has around 100) and be able to achieve monthly releases.
\end{enumerate}

\begin{center}\rule{0.5\linewidth}{0.5pt}\end{center}

\hypertarget{growing-the-contributor-community}{%
\section{Growing the contributor community}\label{growing-the-contributor-community}}

The Neuroconductor platform (https://neuroconductor.org/) ({\textbf{???}}) was started in 2014 at Johns Hopkins to centralize medical image analysis R packages, similar to the effort of Bioconductor (Huber et al., 2015). One of the main strengths of Neuroconductor is the developer community, which involves XX people from YY countries, which contribute to, use, and advertise for Neuroconductor. \textbf{All packages are maintained on the developer's GitHub and mirrored on the Neuroconductor GitHub and releases.}

Comparatively, the project development team is small and but also less
diverse. Moreover, one of the developers (John Muschelli) contributes a total of XX packages to

With 4 releases in the past year (2019), the project has proved its ability to deliver checks and updates to a large number of packages. The stability of those releases has grown over time, which is positive, but shows that improvements to the stability of the network of packages still need to be made. We Neuroconductor would similarly benefit from strengthening and growing its contributor community. (NOT TRANSITION)

We propose to take the following steps to make it easier for new developers and users to contribute to the project.

\hypertarget{developing-an-official-user-guide}{%
\subsection{Developing an official user guide}\label{developing-an-official-user-guide}}

OpenRefine's documentation is currently written in the GitHub wiki associated with the project,2 which is imperfect in many ways: it is hard to
discover, search and browse. There is no simple way to version it or translate it. As a consequence, users tend to write their own tutorials on external
platforms rather than to improve the official documentation.3 Some documentation efforts play the role of official guide in certain communities.4 This
results in a myriad of tutorials, all with a slightly different focus but with often a significant overlap in content, which gradually become outdated as the
software evolves. As a user, this makes it hard to discover the best resources
to learn about certain aspects of the tool.
We propose to migrate the official documentation to a proper platform
such as ReadTheDocs, where it can be versioned and translated natively. By
providing an appealing space to write documentation, we hope to encourage
existing tutorial authors to contribute directly to it. This should help us grow
our contributor base and make it more diverse.

\hypertarget{developing-a-contributor-guide}{%
\subsection{Developing a contributor guide}\label{developing-a-contributor-guide}}

The existing documentation to get started on OpenRefine development does
not help contributors enough. Beyond the migration to a proper documentation platform as mentioned above, we need to give instructions to get started
with the most common Java IDE platforms (Eclipse, IntelliJ, Netbeans) and
provide detailed guidance about common tasks (writing new GREL functions, operations, importers). We can then link to these instructions from the
issues, to encourage prospective contributors to step in.

\hypertarget{revamping-the-core-data-model}{%
\section{revamping the core data model}\label{revamping-the-core-data-model}}

OpenRefine currently uses its own data storage backend. This choice is the
root cause of important limitations in the tool, that we outline below. Changing our data model with these issues in mind is an important investment for
the project to remain relevant in data-intensive fields, such as biology and
medical research.
3.1 Working with large data sets
The requirement to load entire projects in memory to work on them makes
it impractical to work on large datasets such as in genomics medicine. This
requires users to adapt the hard memory limit given to the Java Virtual
Machine depending on the size of their datasets and their available memory,
when they can afford enough RAM.
3.2 Refining collaboratively
OpenRefine is a web-based tool that is designed to be run locally by the user.
Although it can be hosted on a server, it is not designed for collaborative
work. As operations are applied in sequence to the project, working simultaneously on disjoint parts of a dataset is rarely viable. The tool does not
even have a notion of ``user'' which would let it track who performed each
change.
3.3 Analyzing, sharing and reusing workflows
The ability to extract workflows as JSON objects and reapply them on other
projects is a flagship feature of the tool. However, it has serious limitations.
It is hard to understand what a workflow does by looking at its representation in JSON, or even at the project history in the tool itself. There is no
simple way to reorganize a workflow, isolate reusable parts or undo selected
operations buried in the history.
3.4 Running workflows in production
Once a workflow has been created, one could want to run it periodically as
part of a wider pipeline. Although many of OpenRefine's operations can be
easily parallelized {[}19{]}, there is no simple way to run them on data streams
discovered progressively. The scheduling of operations is also naive, as they
are executed in sequence without any time sharing.

\hypertarget{work-plan}{%
\section{work plan}\label{work-plan}}

The first objective, to grow the contributor community, will be tackled by
Owen Stephens. Owen is best placed for this work since he is a long stand-
ing contributor to the project on many aspects, writing documentation, contributing and reviewing code. He also runs many workshops in universities
to teach OpenRefine to researchers and has therefore a deep understanding
of the needs of the user community.
The second objective, to revamp the data model, will be handled by Antonin Delpeuch. As author of some of the research behind the proposed new
architecture {[}20{]}, Antonin is in a good position to implement it in OpenRefine. He has experience with carrying out large refactoring projects on OpenRefine's legacy code and cares deeply about how implementation decisions
will be perceived by the end user.

4.1 Growing the community
Our strategy to attract new contributors and retain existing ones is as follows.
The timescale is given on the assumption of Owen Stephens working part
time on this goal (0.5 full time equivalent).
1. Migrate the existing documentation to a new platform - 1 month
We obtain a versioned and translatable documentation on ReadTheDocs. Existing documentation pages redirect to the new platform to
avoid creating dead links.
2. Write tutorials for prospective contributors - 2 months
We document IDE set up, testing, making good pull requests. We
show example workflows to implement new functions, importers or
exporters.
3. Improve the existing user documentation - 6 months
All operations, facets, expression languages, importers and exporters
are documented.
4. Organize events focused on contribution to the project - 2 months
On the occasion of OpenRefine's 10th anniversary (in 2020), we will
organize local events in targeted communities5
to show how to contribute to the translations, documentation and code. Use the feedback
obtained at such events to inform the tasks above.

4.2 Revamping the data model
The new data model required to address all the issues mentioned in the
previous section is ambitious. We anticipate that implementing it in full is
likely to take significantly longer than a one year project. We decompose the
work needed into packages with intermediate deliverables. We anticipate
one year of full time development should cover packages 1 to 5. The other
goals are mentioned to relate the work to the project's long term milestones.
The timescale is given on the assumption of Antonin Delpeuch working full
time on this goal for the entire duration of the project.

\begin{enumerate}
\def\labelenumi{\arabic{enumi}.}
\tightlist
\item
  Preliminaries: expression languages - 1 month
  Determining which column an expression depends on requires the ability to analyze expressions and extract the variables it reads.
  Performing this analysis requires changing the interface implemented
  by our expression languages (currently GREL, Jython and Clojure by
  default) to allow for such an inspection. This can be done easily for
  GREL but requires more work for other interpreters. Given that we
  already have plans to migrate from Jython to GraalPython, we propose to carry out this migration and implement this analysis using the
  Truffle library (which is used to define interpreters in the Graal framework).
\item
  Column-based operations - 2 months
  We introduce a new interface for row-wise, out of order operations
  reading from a selection of columns. Existing operations which satisfy
  these constraints can implement this interface. Similarly, facet configurations expose which columns they depend on. This new data model
  is based on the representation proposed in {[}20{]}. At this stage the data
  processing backend does not change yet: everything is still held in
  memory.
\item
  New data processing backend - 4 months
  We migrate workflow execution to a dataflow backend such as Apache
  Spark {[}21{]} through Apache Beam's API {[}22{]}. In this version, the tool
  can still execute workflows serialized in the current format, but cannot
  read project data from previous versions.
  We anticipate that the main benefits for end users at this stage are
  mostly felt in memory management (ability to work on datasets which
  do not fit in RAM) and marginal speed improvements. These benefits become important in data-intensive sciences, such as biomedical
  research.
\item
  Workflow visualization - 2 months
  We add diagrammatic representations of the list of operations that
  are implemented, letting users understand better the structure of their
  workflows. This is again based on {[}20{]}.
\item
  Operation reordering - 2 months
  We introduce the possibility of reordering operations in the undo/redo
  history. This feature would only be available for row-wise operations
  working on different columns, implementing the new interface. Reordering could be triggered directly by dragging nodes on the graphical visualization.
\item
  Concurrent operations - if time allows
  We add the possibility of running independent operations concurrently,
  such as reconciling two different columns in parallel. The workflow
  5 existing support 6
  representation should also make it possible to execute different operations on disjoint subsets of rows, but this is likely to be harder to
  expose to the user.
\item
  Partial computations - if time allows
  We return the control to the user before the full computation of longrunning operations terminates. For instance, when reconciling a column, the user can start inspecting the first few reconciliation results
  immediately. They can perform other operations while reconciliation
  is progressing. These other operations are executed immediately even
  if they depend on the reconciliation results. This relies on the streaming capabilities of the underlying execution backend.
\item
  Multi-user support - if time allows
  We add a notion of a ``user'' to keep track of the author of each change
  and make necessary adjustments to the interface and server to let multiple users work on the same project seamlessly
\end{enumerate}

5 existing support
OpenRefine has been supported in 2018 by a 100,000 USD grant from the
Google News Initiative.6 These funds have been fully used for contract work
by the core team in 2018. OpenRefine development is currently carried out
on a voluntary basis.
The research which designed the new architecture for OpenRefine {[}20{]}
was carried out by Antonin Delpeuch as part of his DPhil at the University
of Oxford, which is supported by the Engineering and Physical Sciences
Research Council in the United Kingdom.

\hypertarget{references}{%
\section*{References}\label{references}}
\addcontentsline{toc}{section}{References}

\hypertarget{refs}{}
\begin{cslreferences}
\leavevmode\hypertarget{ref-bioconductor}{}%
Huber, W., Carey, V.J., Gentleman, R., Anders, S., Carlson, M., Carvalho, B.S., Bravo, H.C., Davis, S., Gatto, L., Girke, T., Gottardo, R., Hahne, F., Hansen, K.D., Irizarry, R.A., Lawrence, M., Love, M.I., MacDonald, J., Obenchain, V., Ole's, A.K., Pag`es, H., Reyes, A., Shannon, P., Smyth, G.K., Tenenbaum, D., Waldron, L., Morgan, M., 2015. Orchestrating high-throughput genomic analysis with Bioconductor. Nature Methods 12, 115--121.
\end{cslreferences}


\end{document}

