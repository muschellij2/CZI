\documentclass[]{elsarticle} %review=doublespace preprint=single 5p=2 column
%%% Begin My package additions %%%%%%%%%%%%%%%%%%%
\usepackage[hyphens]{url}



\usepackage{lineno} % add
\providecommand{\tightlist}{%
  \setlength{\itemsep}{0pt}\setlength{\parskip}{0pt}}

\usepackage{graphicx}
\usepackage{booktabs} % book-quality tables
%%%%%%%%%%%%%%%% end my additions to header

\usepackage[T1]{fontenc}
\usepackage{lmodern}
\usepackage{amssymb,amsmath}
\usepackage{ifxetex,ifluatex}
\usepackage{fixltx2e} % provides \textsubscript
% use upquote if available, for straight quotes in verbatim environments
\IfFileExists{upquote.sty}{\usepackage{upquote}}{}
\ifnum 0\ifxetex 1\fi\ifluatex 1\fi=0 % if pdftex
  \usepackage[utf8]{inputenc}
\else % if luatex or xelatex
  \usepackage{fontspec}
  \ifxetex
    \usepackage{xltxtra,xunicode}
  \fi
  \defaultfontfeatures{Mapping=tex-text,Scale=MatchLowercase}
  \newcommand{\euro}{€}
\fi
% use microtype if available
\IfFileExists{microtype.sty}{\usepackage{microtype}}{}
\bibliographystyle{elsarticle-harv}
\usepackage{longtable}
\ifxetex
  \usepackage[setpagesize=false, % page size defined by xetex
              unicode=false, % unicode breaks when used with xetex
              xetex]{hyperref}
\else
  \usepackage[unicode=true]{hyperref}
\fi
\hypersetup{breaklinks=true,
            bookmarks=true,
            pdfauthor={},
            pdftitle={Increasing the Impact of Neuroconductor},
            colorlinks=false,
            urlcolor=blue,
            linkcolor=magenta,
            pdfborder={0 0 0}}
\urlstyle{same}  % don't use monospace font for urls

\setcounter{secnumdepth}{5}
% Pandoc toggle for numbering sections (defaults to be off)


% Pandoc header
\usepackage[margin=2.5cm]{geometry}



\begin{document}
\begin{frontmatter}

  \title{Increasing the Impact of Neuroconductor}
      
  \begin{abstract}
  Over the past 5 years, Neuroconductor has centralized the packages of
  medical image analysis in the R community. As a repository of a wide
  variety of analyses of diseases such as Alzheimer's disease (Ding et al.
  2019) and multiple sclerosis(Valcarcel et al. 2018; E. Sweeney et al.
  2013; E. M. Sweeney et al. 2013), image processing and prediction
  (Tustison et al. 2019; Shrivastav et al. 2018; Polzehl and Tabelow,
  n.d.; Shinohara et al. 2014), image visualization (Maag 2018; Mowinckel
  and Vidal-Piñeiro 2019) and state-of-the-art statistical analyses
  (Vandekar et al. 2019). For Neuroconductor to succeeed for the next 5
  years and beyond, we need to grow its contributor community, and improve
  the stability, usability, and portability of the platform.
  \end{abstract}
  
 \end{frontmatter}

\section{Brief Description of
Neuroconductor}\label{brief-description-of-neuroconductor}

Neuroconductor is a platform for R package submission and repository of
released packages. A user submits a package to Neuroconductor from a
GitHub repository The backend of the platform then runs an initial set
of sanity checks, clones the repository to the Neuroconductor GitHub
repository, then checks the package on the continuous integration Travis
CI (for Linux/OSX) and Appveyor (Windows). Any failures in this package
checks will be reported back to the author (maintainer) of the package
about what failed and what changes need to be made. As all code is on
GitHub, the Neuroconductor team will also send pull requests to fix the
code in many cases. The built packages are then combined every few
months as a release/snapshot of all the passing packages at that time.

\section{Goals}\label{goals}

The main objectives of this proposal are:

\begin{enumerate}
\def\labelenumi{\arabic{enumi}.}
\tightlist
\item
<<<<<<< HEAD
  improve the stability and usability of the current (N=98) packages in Neuroconductor.
=======
  improve the stability and usability of the current (N=98) packages in
  Neuroconductor.
>>>>>>> 769319696f6c1bb558a5b62a86facdb93dba3cd2
\item
  increase the number of packages and community of Neuroconductor's
  contributors and developers by reaching out to seasoned users and
  helping them get involved more closely in the project.
\item
  refactor the core architecture of the Neuroconductor backend platform
  to handle more packages and incorporate user workflows.
\end{enumerate}

Our goals will be reached if:

\begin{enumerate}
\def\labelenumi{\arabic{enumi}.}
\tightlist
\item
<<<<<<< HEAD
  Each package has a vignette (a tutorial/long-form documentation) and the all packages has a code coverage above 50\% (currently 45\% meet this criteria).
=======
  Each package has a vignette (a tutorial/long-form documentation) and
  all packages have a code coverage above 50\% (currently 45\% meet this
  criteria).
>>>>>>> 769319696f6c1bb558a5b62a86facdb93dba3cd2
\item
  We add at least 10 new packages over the course of the project and get
  contributions from 20 existing contributors (out of 26). Contributions
  will be defined as releasing updates to packages through developers
  GitHub repositories or submitting additional packages to the
  collection.
\item
<<<<<<< HEAD
  Scale the Neuroconductor framework to 200 packages (currently has 98) and be able to achieve monthly releases, including releases of Docker images.
\end{enumerate}

\hypertarget{improve-stability-of-neuroconductor-packages}{%
\section{Improve stability of Neuroconductor packages}\label{improve-stability-of-neuroconductor-packages}}

One of the key metrics of package stability is code coverage, which we define as the proportion of the source code that is executed when running tests, examples, or vignettes (tutorials). The \texttt{covr} package (Hester 2019) has functions to report the coverage of a package, which can be reported using services such as CodeCov (https://codecov.io/) or Coveralls (https://coveralls.io/), which is implemented in Neuroconductor.

With 4 releases (https://neuroconductor.org/releases/) in the past year (2019), the project has proved its ability to deliver checks and updates to a large number of packages. The stability of those releases has grown over time, which is positive, but shows that improvements to the stability of the network of packages still need to be made.

John Muschelli will send a series of pull requests to packages that do not have code coverage over 50\%. These pull requests will try to increase the code coverage to necessary level, but moreso provide the groundwork for maintainers to create their own tests (especially those for edge cases). Additionally, these pull requests will create vignettes of how to use the package if none exist. These vignettes will have intentional errors and breaks so that developers must change these before submitting to Neuroconductor. Documentation on increasing code coverage will be created to more formalize this process and help developers (see the next section).

By the end of the funding year, all current packages must meet these requirements to stay within the Neuroconductor repository.

\hypertarget{growing-the-contributor-community}{%
\section{Growing the contributor community}\label{growing-the-contributor-community}}

The Neuroconductor platform (https://neuroconductor.org/) (Muschelli et al. 2018) was started in 2014 at Johns Hopkins to centralize medical image analysis R packages, similar to the effort of Bioconductor (Huber et al. 2015). One of the main strengths of Neuroconductor is the developer community, which involves 26 maintainers residing in 5 countries, which contribute to, use, and advertise for Neuroconductor. Our plan to strengthen the community is to host a conference call/meetup twice a year to discuss developments with the maintainers, including requesting feedback for ways to improve the platform, the submission process, feature requests, or increasing user/developer engagement. With respec to users, we have had users in over 48 countries download packages from Neuroconductor.

Comparatively, the project development team is small and but also less diverse. The team is currently all white males, 2 US-born and 2 Romanian-born. Moreover, one of the developers (John Muschelli) contributes a total of 48 packages to the project. Many of these packages (18) are data packages, which include image templates or example data sets for analysis, requiring little maintenance. We would like to increase the number of developers, especially those from different countries, races, and sex. Currently, out of the 26 maintainers, only 6 are female (23\%).

We propose to take the following steps to make it easier for new developers and users to contribute to the project.

\hypertarget{developing-a-code-coverage-user-guide}{%
\subsection{Developing a code coverage user guide}\label{developing-a-code-coverage-user-guide}}

We have created documentation about what changes or checks are done to packages when submitting to Neuroconductor (https://neuroconductor.org/tutorials/continuous\_integration) but no clear documentation of the expectations of the package.
=======
  Scale the Neuroconductor framework to 200 packages (currently has 98)
  and be able to achieve monthly releases, including releases of Docker
  images.
\end{enumerate}

\section{Improve stability of Neuroconductor
packages}\label{improve-stability-of-neuroconductor-packages}

One of the key metrics of package stability is code coverage, which we
define as the proportion of the source code that is executed when
running tests, examples, or vignettes (tutorials). The \texttt{covr}
package (Hester 2019) has functions to report the coverage of a package,
which can be reported using services such as CodeCov
(https://codecov.io/) or Coveralls (https://coveralls.io/), which is
implemented in Neuroconductor.

With 4 releases in the past year (2019), the project has proved its
ability to deliver checks and updates to a large number of packages. The
stability of those releases has grown over time, which is positive, but
shows that improvements to the stability of the network of packages
still need to be made.

John Muschelli will send a series of pull requests to packages that do
not have code coverage over 50\%. These pull requests will try to
increase the code coverage to necessary level, but moreso provide the
groundwork for maintainers to create their own tests (especially those
for edge cases). Additionally, these pull requests will create vignettes
of how to use the package if none exist. These vignettes will have
intentional errors and breaks so that developers must change these
before submitting to Neuroconductor. Documentation on increasing code
coverage will be created to more formalize this process and help
developers (see the next section).

By the end of the funding year, all current packages must meet these
requirements to stay within the Neuroconductor repository.

\section{Growing the contributor
community}\label{growing-the-contributor-community}

The Neuroconductor platform (https://neuroconductor.org/) (Muschelli et
al. 2018) was started in 2014 at Johns Hopkins to centralize medical
image analysis R packages, similar to the effort of Bioconductor (Huber
et al. 2015). One of the main strengths of Neuroconductor is the
developer community, which involves 26 maintainers residing in 5
countries, which contribute to, use, and advertise for Neuroconductor.
Our plan to strengthen the community is to host a conference call/meetup
twice a year to discuss developments with the maintainers, including
requesting feedback for ways to improve the platform, the submission
process, feature requests, or increasing user/developer engagement. With
respect to users, we have had users in over 48 countries download
packages from Neuroconductor.

Comparatively, the project development team is small and but also less
diverse. The team is currently all white males, 2 US-born and 2
Romanian-born. Moreover, one of the developers (John Muschelli)
contributes a total of 48 packages to the project. Many of these
packages (18) are data packages, which include image templates or
example data sets for analysis, requiring little maintenance. We would
like to increase the number of developers, especially those from
different countries, races, and sex. Currently, out of the 26
maintainers, only 6 are female (23\%).

We propose to take the following steps to make it easier for new
developers and users to contribute to the project.

\subsection{Developing a code coverage user
guide}\label{developing-a-code-coverage-user-guide}

We have created documentation about what changes or checks are done to
packages when submitting to Neuroconductor
(https://neuroconductor.org/tutorials/continuous\_integration) but no
clear documentation of the expectations of the package.
>>>>>>> 769319696f6c1bb558a5b62a86facdb93dba3cd2

We propose to create:

\begin{enumerate}
\def\labelenumi{\arabic{enumi}.}
\tightlist
\item
  Formal package guidelines that mirror closely, but not exactly, those
  from Bioconductor
  (https://www.bioconductor.org/developers/package-guidelines).
\item
  Examples of how to create unit tests and increase code coverage in
  Neuroconductor packages.
\item
  A contributor guide of how to create workflows and submit them to the
  tutorial area of Neuroconductor
  (https://neuroconductor.org/tutorials).
\end{enumerate}

<<<<<<< HEAD
The last point is different than vignettes in a subtle way. As we require all vignettes in Neuroconductor to be built on Travis CI or Appveyor, the data must be accessible programmatically, enclosed in the package, or use data from other packages. Package maintainers and developers may make full pipelines or example analyses that may be too computation-heavy for vignettes, interact with data sources that are not open, or use many packages that are not desirable dependencies for that package. The tutorial area is for these analyses, but it is currently unclear how to create or submit these, thus we need a contributor guide.

\hypertarget{adding-new-packages}{%
\subsection{Adding new packages}\label{adding-new-packages}}

There are a number of packages that exist in R for imaging, but are only hosted on GitHub or both GitHub/CRAN. We reach out to these authors whenever we find those packages, but the effort is not always formalized. Thus, we use R Documetation (https://www.rdocumentation.org/), which aggregates packages from all major R repositories to identify packages that are imaging-related and reach out to maintainers. Along with the every-other-month call above, we believe these steps will increase the number (and hopefully the diversity) of developers and the number of contributions of developers. We will count the number of package updates in our backend.

\hypertarget{hands-on-tutorial}{%
\subsection{Hands on tutorial}\label{hands-on-tutorial}}

Either on site in Baltimore, at a remote site, or at an imaging conference, one in-person, hands-on tutorial will be given by John Muschelli. This tutorial will be used to create an online course that will be either hosted on LeanPub or Coursera.

\hypertarget{refactor-the-neuroconductor-framework}{%
\section{Refactor the Neuroconductor framework}\label{refactor-the-neuroconductor-framework}}

\hypertarget{portability-of-neuroconductor}{%
\subsection{Portability of Neuroconductor}\label{portability-of-neuroconductor}}

Though most of the checking of packages is done on cloud-based continuous integration services, the backend is not in a true cloud platform, therefore not as portable as we would like. Currently, the Neuroconductor backend is hosted on a server at Johns Hopkins University with Drupal, PHP, and Centos 6. Over time, we have upgraded the backend to new operating systems, Drupal and PHP versions, and added additional third-party software to the server. Though this has worked for our current needs, we wish to build the backend on a cloud-based server, likely Amazon Web Services (AWS).

To address the portability of the system, we will create a custom AWS image that can be used to spool up a backend. This image will be versioned and backed up. Thus, if we wish to create another centralized platform for R packages (such as for wearable devices), this should be possible without much work.

\hypertarget{ease-of-use-of-neuroconductor}{%
\subsection{Ease of Use of Neuroconductor}\label{ease-of-use-of-neuroconductor}}

We will also create a series of Docker images that have a large percentage of the Neuroconductor packages installed, including open-source third party medical imaging software such as AFNI and ANTs (Cox 1996; Avants, Tustison, and Song 2009) and other-licensed software, such as FSL (Jenkinson et al. 2012). These images will increase the ease of use for a number of users, especially those on Windows, as many imaging software packages may not work on Windows (though almost all R packages should). An example use case is providing these Docker images for students in our ``Neurohacking in R'' course (https://www.coursera.org/learn/neurohacking), which currently provides a static virtual machine for use.

\hypertarget{scaling-neuroconductor}{%
\subsection{Scaling Neuroconductor}\label{scaling-neuroconductor}}

Though Neuroconductor can handle the 98 packages, we wish to grow the community and number of packages. Whenever a new package is submitted, that package must be checked, along with any packages that depend on that package. That creates a large number of continuous integration (CI) jobs. Thus, increasing the number of packages will require either a) more concurrent jobs on cloud systems, or b) the setup of servers for our needs. Now, we believe the CI systems we employ can handle our needs, but we need to expand above the free, single-user plans, while keeping the dependency structure bookkeeping on our server side.

\hypertarget{work-plan}{%
\section{Work plan}\label{work-plan}}

The first objective to improve stability of Neuroconductor packages, John Muschelli is best placed to do this as a developer of R packages for over 10 years and contributor to many Neuroconductor packages already.

The second objective, to grow the contributor community, will be tackled by
John Muschelli, Brian Caffo, and Ciprian Crainiceanu. Ciprian and Brian are best placed to grow the community due to their large networks in the statistical and imaging communities. Each has also organized meetings of many different parties, such as full conferences (ENAR - https://enar.org/).

The third objective, to scale the Neuroconductor platform, will be handled by Adrian (Adi) Gherman. As the main developer on the Neuroconductor backend, Adi is in a good position to implement the changes. He has experience with Drupal, PHP, and Docker, including carrying out large changes to the Neuroconductor system.

\hypertarget{improve-stability}{%
\subsection{Improve Stability}\label{improve-stability}}
=======
The last point is different than vignettes in a subtle way. As we
require all vignettes in Neuroconductor to be built on Travis CI or
Appveyor, the data must be accessible programmatically, enclosed in the
package, or use data from other packages. Package maintainers and
developers may make full pipelines or example analyses that may be too
computation-heavy for vignettes, interact with data sources that are not
open, or use many packages that are not desirable dependencies for that
package. The tutorial area is for these analyses, but it is currently
unclear how to create or submit these, thus we need a contributor guide.

\subsection{Adding new packages}\label{adding-new-packages}

There are a number of packages that exist in R for imaging, but are only
hosted on GitHub or both GitHub/CRAN. We reach out to these authors
whenever we find those packages, but the effort is not always
formalized. Thus, we use R Documetation
(https://www.rdocumentation.org/), which aggregates packages from all
major R repositories to identify packages that are imaging-related and
reach out to maintainers. Along with the every-other-month call above,
we believe these steps will increase the number (and hopefully the
diversity) of developers and the number of contributions of developers.
We will count the number of package updates in our backend.

\subsection{Hands on tutorial}\label{hands-on-tutorial}

Either on site in Baltimore, at a remote site, or at an imaging
conference, one in-person, hands-on tutorial will be given by John
Muschelli. This tutorial will be used to create an online course that
will be either hosted on LeanPub or Coursera.

\section{Refactor the Neuroconductor
framework}\label{refactor-the-neuroconductor-framework}

\subsection{Portability of
Neuroconductor}\label{portability-of-neuroconductor}

Though most of the checking of packages is done on cloud-based
continuous integration services, the backend is not in a true cloud
platform, therefore not as portable as we would like. Currently, the
Neuroconductor backend is hosted on a RedHat 6.10 server at Johns
Hopkins University with the backend consisting of Drupal 8 CMS framework
that relies on PHP 5.3, MySQL 14.14 and Apache 2.4. Over time, we have
upgraded the backend to new operating systems, Drupal and PHP versions,
and added additional third-party software to the server. Though this has
worked for our current needs, we wish to migrate the backend to a
cloud-based server, likely Amazon Web Services (AWS).

The backend code is open, located at
https://github.com/adigherman/neuroconductor. To address the portability
of the system, we will create a custom AWS image that can be used to
spool up a backend. This image will be versioned and backed up. Thus, if
we wish to create another centralized platform for R packages (such as
for wearable devices), this should be possible without much work.

\subsection{Ease of Use of
Neuroconductor}\label{ease-of-use-of-neuroconductor}

We will also create a series of Docker images that have a large
percentage of the Neuroconductor packages installed, including
open-source third party medical imaging software such as AFNI and ANTs
(Cox 1996; Avants, Tustison, and Song 2009) and other-licensed software,
such as FSL (Jenkinson et al. 2012). These images will increase the ease
of use for a number of users, especially those on Windows, as many
imaging software packages may not work on Windows (though almost all R
packages should). An example use case is providing these Docker images
for students in our ``Neurohacking in R'' course
(https://www.coursera.org/learn/neurohacking), which currently provides
a downloadable virtual machine image for use.

\subsection{Scaling Neuroconductor}\label{scaling-neuroconductor}

Though Neuroconductor can handle the 98 packages, we wish to grow the
community and number of packages. Whenever a new package is submitted,
that package must be checked, along with any packages that depend on
that package. That creates a large number of continuous integration (CI)
jobs. Thus, increasing the number of packages will require either a)
more concurrent jobs on cloud systems, or b) the setup of servers for
our needs. Now, we believe the CI systems we employ can handle our
needs, but we need to expand above the free, single-user plans, while
keeping the dependency structure bookkeeping on our server side.

\section{Work plan}\label{work-plan}

The first objective is to improve stability of Neuroconductor packages.
John Muschelli is best placed to do this as a developer of R packages
for over 10 years and contributor to many Neuroconductor packages
already.

The second objective, to grow the contributor community, will be tackled
by John Muschelli, Brian Caffo, and Ciprian Crainiceanu. Ciprian and
Brian are best placed to grow the community due to their large networks
in the statistical and imaging communities. Each has also organized
meetings of many different parties, such as full conferences (ENAR -
https://enar.org/).

The third objective, to scale the Neuroconductor platform, will be
handled by Adrian (Adi) Gherman. As the main developer on the
Neuroconductor backend, Adi is in a good position to implement the
changes. He has experience with Drupal, PHP, and Docker, including
carrying out large changes to the Neuroconductor system.

\subsection{Improve Stability}\label{improve-stability}
>>>>>>> 769319696f6c1bb558a5b62a86facdb93dba3cd2

The timescale is given on the assumption of John Muschelli working part
time on this goal (0.35 full time equivalent). Our strategy to improve
the stability of current packages is as follows:

\begin{enumerate}
\def\labelenumi{\arabic{enumi}.}
\tightlist
\item
  For the 54 with insufficient code coverages, make pull requests to the
  packages (approximately 10 per month) - 6 months.\\
\item
  Write tutorials for prospective contributors/developers on increasing
  code coverage and improved submissions - 2 months
\item
  Improve the existing user documentation on https://neuroconductor.org/
  - 3 months
\end{enumerate}

\subsection{Growing the contributor
community}\label{growing-the-contributor-community-1}

The timescale is given on the assumption of Ciprian Crainiceanu and
Brian Caffo working part time on this goal (0.05 full time equivalent)
and above time from John Muschelli.

\begin{enumerate}
\def\labelenumi{\arabic{enumi}.}
\tightlist
\item
  Set up a conference call once every 2 months - 1 month
\item
  Reach out to R-ladies Baltimore and other R-ladies group to discuss
  Neuroconductor to increase diversity in the user base - 2 months
\item
  Organize sessions related to Neuroconductor and open source software
  at imaging and statistics conferences - 1 month
\item
  Provide one hands-on tutorial - 3 months
\item
  Create a survery for developers to get more detailed information, such
  as identified sex, race, age, and other factors such as education
  attained and target analysis diseases - 2 months
\end{enumerate}

\subsection{Refactor the Neuroconductor
framework}\label{refactor-the-neuroconductor-framework-1}

The timescale is given on the assumption of Adrian Gherman working part
time on this goal (0.5 full time equivalent).

\begin{enumerate}
\def\labelenumi{\arabic{enumi}.}
\tightlist
\item
  Setting up AWS systems and testing - 1 month
\item
  Upgrading the Neuroconductor system to a newer OS version and most
  up-to-date Drupal/PHP/MySQL/Security - 2 months
\item
  Automating the creation of Docker images - 2 months
\item
  Implements new checks for code coverage and vignettes - 2 months
\end{enumerate}

\section{Potential areas for growth}\label{potential-areas-for-growth}

Though we wish to stabilize the Neuroconductor system, we wish to
integrate analyses of imaging from Neuroconductor and ``omics'' from
Bioconductor. This integration is challenging due to the size of the
data and heterogeneity of the sources, amongst many other reasons. For a
review, see Antonelli et al. (2019). For example, the \texttt{limmi}
package (https://github.com/muschellij2/limmi) is an attempt to coerce
functional magnetic resonance imaging into a testing framework that both
works with \texttt{limma} package (Ritchie et al. 2015) and into the
\texttt{SummarizedExperiment} format, which both are fundamental
elements from Bioconductor (Huber et al. 2015).

\section{Existing support}\label{existing-support}

We are currently applying for AWS credits in addition to the funding for
computing here. Neuroconductor has been supported by General Funds from
the Biostatistics Department at Johns Hopkins Bloomberg School of Public
Health. NIH Grants XXX currently have a small amount of funding to
support development of packages for the system, which have been used for
package and backend development.

This Grant was generated with all materials at
https://github.com/muschellij2/CZI and can be downloaded from
https://johnmuschelli.com/CZI/index.pdf.

\section*{References}\label{references}
\addcontentsline{toc}{section}{References}

\hypertarget{refs}{}
\hypertarget{ref-antonelli2019integrating}{}
Antonelli, Laura, Mario Rosario Guarracino, Lucia Maddalena, and Mara
Sangiovanni. 2019. ``Integrating Imaging and Omics Data: A Review.''
\emph{Biomedical Signal Processing and Control} 52. Elsevier: 264--80.

\hypertarget{ref-avants2009advanced}{}
Avants, Brian B, Nick Tustison, and Gang Song. 2009. ``Advanced
Normalization Tools (ANTS).'' \emph{Insight J} 2: 1--35.

\hypertarget{ref-afni}{}
Cox, Robert W. 1996. ``AFNI: Software for Analysis and Visualization of
Functional Magnetic Resonance Neuroimages.'' \emph{Computers and
Biomedical Research} 29 (3). Elsevier: 162--73.

\hypertarget{ref-ding2019improved}{}
Ding, T, AD Cohen, EE O'Connor, HT Karim, A Crainiceanu, J Muschelli, O
Lopez, et al. 2019. ``An Improved Algorithm of White Matter
Hyperintensity Detection in Elderly Adults.'' \emph{NeuroImage:
Clinical}. Elsevier, 102151.

\hypertarget{ref-covr}{}
Hester, Jim. 2019. \emph{covr: Test Coverage for Packages}.
\url{https://CRAN.R-project.org/package=covr}.

\hypertarget{ref-bioconductor}{}
Huber, W., V. J. Carey, R. Gentleman, S. Anders, M. Carlson, B. S.
Carvalho, H. C. Bravo, et al. 2015. ``Orchestrating High-Throughput
Genomic Analysis with Bioconductor.'' \emph{Nature Methods} 12 (2):
115--21.
\url{http://www.nature.com/nmeth/journal/v12/n2/full/nmeth.3252.html}.

\hypertarget{ref-fsl}{}
Jenkinson, Mark, Christian F. Beckmann, Timothy E. J. Behrens, Mark W.
Woolrich, and Stephen M. Smith. 2012. ``FSL.'' \emph{NeuroImage} 62 (2):
782--90.
doi:\href{https://doi.org/10.1016/j.neuroimage.2011.09.015}{10.1016/j.neuroimage.2011.09.015}.

\hypertarget{ref-maag2018gganatogram}{}
Maag, Jesper LV. 2018. ``gganatogram: An R Package for Modular
Visualisation of Anatograms and Tissues Based on ggplot2.''
\emph{F1000Research} 7. Faculty of 1000 Ltd.

\hypertarget{ref-mowinckel2019visualisation}{}
Mowinckel, Athanasia M, and Didac Vidal-Piñeiro. 2019. ``Visualisation
of Brain Statistics with R-Packages Ggseg and Ggseg3d.'' \emph{arXiv
Preprint arXiv:1912.08200}.

\hypertarget{ref-neuroconductor}{}
Muschelli, John, Adrian Gherman, Jean-Philippe Fortin, Brian Avants,
Brandon Whitcher, Jonathan D Clayden, Brian S Caffo, and Ciprian M
Crainiceanu. 2018. ``Neuroconductor: An R Platform for Medical Imaging
Analysis.'' \emph{Biostatistics}, kxx068.
doi:\href{https://doi.org/10.1093/biostatistics/kxx068}{10.1093/biostatistics/kxx068}.

\hypertarget{ref-polzehlmagnetic}{}
Polzehl, Jörg, and Karsten Tabelow. n.d. ``Magnetic Resonance Brain
Imaging.'' Springer.

\hypertarget{ref-limma}{}
Ritchie, Matthew E, Belinda Phipson, Di Wu, Yifang Hu, Charity W Law,
Wei Shi, and Gordon K Smyth. 2015. ``limma Powers Differential
Expression Analyses for RNA-Sequencing and Microarray Studies.''
\emph{Nucleic Acids Research} 43 (7): e47.
doi:\href{https://doi.org/10.1093/nar/gkv007}{10.1093/nar/gkv007}.

\hypertarget{ref-shinohara2014statistical}{}
Shinohara, Russell T, Elizabeth M Sweeney, Jeff Goldsmith, Navid Shiee,
Farrah J Mateen, Peter A Calabresi, Samson Jarso, et al. 2014.
``Statistical Normalization Techniques for Magnetic Resonance Imaging.''
\emph{NeuroImage: Clinical} 6. Elsevier: 9--19.

\hypertarget{ref-shrivastav2018classification}{}
Shrivastav, Kumar Dron, Ankan Mukherjee Das, Harpreet Singh, Priya
Ranjan, and Rajiv Janardhanan. 2018. ``Classification of Colposcopic
Cervigrams Using EMD in R.'' In \emph{International Symposium on Signal
Processing and Intelligent Recognition Systems}, 298--308. Springer.

\hypertarget{ref-sweeney2013oasis}{}
Sweeney, Elizabeth M, Russell T Shinohara, Navid Shiee, Farrah J Mateen,
Avni A Chudgar, Jennifer L Cuzzocreo, Peter A Calabresi, Dzung L Pham,
Daniel S Reich, and Ciprian M Crainiceanu. 2013. ``OASIS Is Automated
Statistical Inference for Segmentation, with Applications to Multiple
Sclerosis Lesion Segmentation in MRI.'' \emph{NeuroImage: Clinical} 2.
Elsevier: 402--13.

\hypertarget{ref-sweeney2013automatic}{}
Sweeney, EM, RT Shinohara, CD Shea, DS Reich, and Ciprian M Crainiceanu.
2013. ``Automatic Lesion Incidence Estimation and Detection in Multiple
Sclerosis Using Multisequence Longitudinal MRI.'' \emph{American Journal
of Neuroradiology} 34 (1). Am Soc Neuroradiology: 68--73.

\hypertarget{ref-tustison2019longitudinal}{}
Tustison, Nicholas J, Andrew J Holbrook, Brian B Avants, Jared M
Roberts, Philip A Cook, Zachariah M Reagh, Jeffrey T Duda, et al. 2019.
``Longitudinal Mapping of Cortical Thickness Measurements: An
Alzheimer's Disease Neuroimaging Initiative-Based Evaluation Study.''
\emph{Journal of Alzheimer's Disease} 71 (1). IOS Press: 165--83.

\hypertarget{ref-valcarcel2018dual}{}
Valcarcel, Alessandra M, Kristin A Linn, Fariha Khalid, Simon N
Vandekar, Shahamat Tauhid, Theodore D Satterthwaite, John Muschelli,
Melissa Lynne Martin, Rohit Bakshi, and Russell T Shinohara. 2018. ``A
Dual Modeling Approach to Automatic Segmentation of Cerebral T2
Hyperintensities and T1 Black Holes in Multiple Sclerosis.''
\emph{NeuroImage: Clinical} 20. Elsevier: 1211--21.

\hypertarget{ref-vandekar2019robust}{}
Vandekar, Simon N, Theodore D Satterthwaite, Cedric H Xia, Azeez
Adebimpe, Kosha Ruparel, Ruben C Gur, Raquel E Gur, and Russell T
Shinohara. 2019. ``Robust Spatial Extent Inference with a Semiparametric
Bootstrap Joint Inference Procedure.'' \emph{Biometrics} 75 (4). Wiley
Online Library: 1145--55.


\end{document}


